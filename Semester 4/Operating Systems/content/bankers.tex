\section{Bankers' Algorithm}
\label{sec:bankers}

\subsection{Aim}
Implementation of the Banker's Algorithm.

\subsection{Theory}
The Banker's Algorithm is a resource allocation and deadlock
avoidance algorithm that tests for safety by simulating the
allocation for predetermined maximum possible amounts of all
resources, and then makes an "s-state" check to test for possible
activities, before deciding whether allocation should be allowed to continue.
It was proposed by Edsger Dijkstra in 1965 and published in 1968.

\subsection{Problem Statement}

The following is a description of the resources used in the Banker's Algorithm.

\begin{itemize}
	\item \textbf{Resource Type} - A resource type is a type of resource that can be allocated to a process. For example, a computer system may have three resource types: CPU, I/O, and Memory\
	\item \textbf{Resource} - A resource is an instance of a resource type. For example, a computer system may have 10 CPU resources, 5 I/O resources, and 100 Memory resources.
	\item \textbf{Resource Vector} - A resource vector is a vector of resources.
	\item \textbf{Maximum Claim} - The maximum claim of a process is the maximum number of resources of each type that the process may request. For example, a process may have a maximum claim of 10 CPU resources, 5 I/O resources, and 100 Memory resources.
	\item \textbf{Allocation} - The allocation of a process is the number of resources of each type that have been allocated to the process. For example, a process may have an allocation of 2 CPU resources, 1 I/O resource, and 50 Memory resources.
	\item \textbf{Need} - The need of a process is the number of resources of each type that the process still needs. For example, a process may have a need of 8 CPU resources, 4 I/O resources, and 50 Memory resources.
\end{itemize}

\subsection{Algorithm}

\subsubsection{Initialization}

The Banker's Algorithm is initialized with the following:

\begin{itemize}
	\item A resource vector \textbf{available} that contains the number of available resources of each type.
	\item A maximum claim matrix \textbf{max} that contains the maximum number of resources of each type that each process may request.
	\item An allocation matrix \textbf{allocation} that contains the number of resources of each type that have been allocated to each process.
\end{itemize}

The need matrix \textbf{need} is then calculated as follows:

\begin{equation}
	\textbf{need} = \textbf{max} - \textbf{allocation}
\end{equation}

\subsubsection{Safe State}

\begin{algorithm}
	\caption{Safe State}
	\label{alg:safe_state}
	\begin{algorithmic}
		\State \textbf{Input:} \textbf{available}, \textbf{max}, \textbf{allocation}
		\State \textbf{Output:} \textbf{true} if \textbf{available} can satisfy \textbf{need}, \textbf{false} otherwise
		\State \textbf{Initialize:} \textbf{work} = \textbf{available}, \textbf{finish} = \textbf{false}
		\While{there exists a process \textbf{p} such that \textbf{finish[p]} is \textbf{false}}
		\If{\textbf{need[p]} $\leq$ \textbf{work}}
		\State \textbf{work} = \textbf{work} + \textbf{allocation[p]}
		\State \textbf{finish[p]} = \textbf{true}
		\EndIf
		\EndWhile
		\State \textbf{Return:} \textbf{true} if \textbf{finish} is \textbf{true} for all processes, \textbf{false} otherwise
	\end{algorithmic}
\end{algorithm}

\pagebreak
\subsubsection{Request Resources}

The request resources algorithm is as follows:

\begin{algorithm}
	\caption{Request Resources}
	\label{alg:request_resources}
	\begin{algorithmic}
		\Require A process $P$ and a resource vector $request$
		\Ensure The request is either granted or denied
		\State $need = max - allocation$
		\If{$request > need$}
		\State Deny request
		\Else
		\If{$request > available$}
		\State Deny request
		\Else
		\State $available = available - request$
		\State $allocation = allocation + request$
		\State $need = need - request$
		\If{Safe State}
		\State Grant request
		\Else
		\State Deny request
		\EndIf
		\EndIf
		\EndIf
	\end{algorithmic}
\end{algorithm}

\subsection{Implementation}

\inputminted[fontsize=\footnotesize,autogobble]{c}{code/bankers.c}

\subsection{Output}
\begin{lstlisting}[style=output]
Following is the SAFE Sequence
P1 -> P3 -> P4 -> P0 -> P2
\end{lstlisting}
