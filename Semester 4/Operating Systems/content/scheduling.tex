\section{CPU Scheduling: FCFS}
\label{sec:cpu-scheduling}

\subsection{Aim}
Implementation of CPU scheduling in a first come first serve (FCFS) manner.

\subsection{Implementation}

\inputminted[fontsize=\footnotesize,autogobble]{c}{code/fcfs.c}

\subsection{Output}
The scheduler selects the process that has been waiting the longest.
\begin{lstlisting}[style=output]
Processes  Burst Time   Waiting Time    Turn Around Time
1	   10		0		10
2	   5		10		15
3	   8		15		23
Average waiting time = 8.333333
Average turn around time = 16.000000
\end{lstlisting}

\pagebreak
\section{CPU Scheduling: Priority}
\label{sec:cpu-scheduling-priority}

\subsection{Aim}
Implementation of CPU scheduling using a ``priority'' based approach using assigned ranks/priority algorithms.

\subsection{Theory}
A major problem with priority scheduling is indefinite blocking or starvation. A solution to the problem of indefinite blockage of the low-priority process is aging. Aging is a technique of gradually increasing the priority of processes that wait in the system for a long period of time.

\subsection{Implementation}

\inputminted[fontsize=\footnotesize,autogobble]{c}{code/priority.c}

\subsection{Output}
\begin{lstlisting}[style=output]
Order in which processes gets executed
2 3 4 1 
Process         BurstTime       WaitingTime     TurnAroundTime
2		19		0		19
3		27		19		46
4		25		46		71
1		100		71		171
Average waiting time = 34.000000
Average turn around time = 76.750000
\end{lstlisting}

\pagebreak
\section{CPU Scheduling: SJF}

\subsection{Aim}
Implementation of CPU scheduling using the shortest job first (SJF) approach.

\subsection{Algorithm}

\begin{algorithm}
	\caption{Shortest Job First}
	\label{alg:sjf}
	\begin{algorithmic}
		\State \textbf{Input:} $n$ processes with their burst times $bt_i$ and arrival times $at_i$
		\State \textbf{Output:} The order in which the processes are executed
		\State $t \gets 0$ \Comment{current time}
		\State $i \gets 0$ \Comment{current process}
		\State $bt \gets \{bt_1, \dots, bt_n\}$ \Comment{burst times}
		\State $at \gets \{at_1, \dots, at_n\}$ \Comment{arrival times}
		\State $bt' \gets \{bt_1, \dots, bt_n\}$ \Comment{remaining burst times}
		\State $wt \gets \{0, \dots, 0\}$ \Comment{waiting times}
		\State $tat \gets \{0, \dots, 0\}$ \Comment{turnaround times}
		\While{$i < n$}
			\If{$at_i \leq t$ and $bt'_i > 0$} \Comment{process is ready}
				\State $bt'_i \gets bt'_i - 1$
				\State $t \gets t + 1$
				\If{$bt'_i = 0$}
					\State $tat_i \gets t - at_i$
					\State $wt_i \gets tat_i - bt_i$
					\State $i \gets i + 1$
				\EndIf
				\Else
					\State $t \gets t + 1$
			\EndIf
		\EndWhile
		\State \textbf{Return:} $wt$ and $tat$
	\end{algorithmic}
\end{algorithm}

\subsection{Implementation}

\inputminted[fontsize=\footnotesize,autogobble]{c}{code/sjf.c}

\subsection{Output}
\begin{lstlisting}[style=output]
Enter number of process: 5
Enter Burst Time:
P[1]: 1
P[2]: 2
P[3]: 3
P[4]: 4
P[5]: 5
Process         BurstTime       WaitingTime     TurnAroundTime
1               1               0               1
2               2               1               3
3               3               3               6
4               4               6               10
5               5               10              15
Average Waiting Time: 4
Average Turnaround Time: 7
\end{lstlisting}
