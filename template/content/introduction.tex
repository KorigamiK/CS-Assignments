\section{Introduction}
This is an example of the \texttt{pracjourn} document class.
Articles should always start with a section title, which will usually be `Introduction' or somesuch.

This document demonstrates the features of the document class.
Here are some logos and abbreviations:
\note{Tuned for Palatino, but refinements welcomed.}
\TeX, \pdfTeX, \BibTeX,\\ \MF, \MP, \LaTeX, \LaTeXe,
\mbox{\ConTeXt}, \pdfLaTeX, \XeTeX, \PracTeX, \TPJ, \PS.

On the following pages are shown the document divisions and list environments.

\section{Section headings}
\itshape The fact is that his precocity in vice was awful. At five months of age he
used to get into such passions that he was unable to articulate. At six
months, I caught him gnawing a pack of cards. At seven months he was in
the constant habit of catching and kissing the female babies. At eight
months he peremptorily refused to put his signature to the Temperance
pledge.
\note{Text from Edgar Allen Poe's `Never bet the Devil your head'.}

\subsection{Subsection}

Poverty was another vice which the peculiar physical deficiency of
Dammit's mother had entailed upon her son. He was detestably poor, and
this was the reason, no doubt, that his expletive expressions about
betting, seldom took a pecuniary turn. I will not be bound to say that I
ever heard him make use of such a figure of speech as ``I'll bet you a
dollar.'' It was usually ``I'll bet you what you please,'' or ``I'll bet you
what you dare,'' or ``I'll bet you a trifle,'' or else, more significantly
still, \emph{``I'll bet the Devil my head.''}

\subsubsection{Subsubsection}


\upshape Subsubsections are in the same font size as the body text, so I guess
it's possibly a little small. I doubt many people really will wish to
use a subsubsection in a journal article. But if they wish, there's
nothing stopping 'em.

\note{Another footnote test.
	This one has to be really long to see what happens
	when a footnote extends over more than one line.
	I.e., two or more.}

\begin{figure}[!ht]
	\centering test
	\caption{test}
\end{figure}

\paragraph{Paragraphs} Named paragraphs are very similar to description lists; however, these may be useful more \emph{multiple} paragraphs, whereas description lists work better with only a single paragraph per item.

\subparagraph{Subparagraphs} For those who want them\dots.

\section{Lists}
An itemized list:
\begin{itemize}
	\item one,
	\item two, and
	\item three.
	      A nested itemised list:
	      \begin{itemise}
		      \item one,
		      \item two, and
		      \item three.
	      \end{itemise}
\end{itemize}
An enumerated list:
\begin{enumerate}
	\item one,
	\item two, and
	\item three.
\end{enumerate}
A description list:
\begin{description}
	\item [one] the first,
	\item [two] the second, and
	\item [three] the third. A nested description list:
	      \begin{description}
		      \item [one] the first,
		      \item [two] the second, and
		      \item [three] the third.
	      \end{description}
\end{description}
