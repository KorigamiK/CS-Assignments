\documentclass{korigamik}

\usepackage{minted}
\usepackage{xcolor}
\usepackage{listings}
\usepackage{algpseudocode}
\usepackage{algorithm}
\usepackage{enumitem}

\title{Advanced Computer Networks}
\titlelabel{Lab Report}
\bottomnote{Department of Computer Science \& Engineering}
\author{Kushagra Lakhwani}
\rollno{2021UCI8036}
\semester{5th}
\course{CICPC16}
\logoimage{res/NSUT.png}{0.6}{Netaji Subhas University \\ of Technology}
\header{}

\colorlet{mycoolgray}{gray!40}
\lstdefinestyle{output}{
	numbers=none, % where to put the line-numbers
	numberstyle=\tiny, % the size of the fonts that are used for the line-numbers
	backgroundcolor=\color{darkgray},
	basicstyle=\ttfamily\color{white}\footnotesize,
	captionpos=b, % sets the caption-position to bottom
	breaklines=true, % sets automatic line breaking
	breakatwhitespace=false,
	keywordstyle=\color{white}\bfseries,
	language=bash,
}

\usepackage{tocloft}
\renewcommand{\cftsecafterpnum}{\qquad\rule{2cm}{0.4pt}}
\setcounter{tocdepth}{1}

\begin{document}

\maketitle
\newpage

\center\textbf{\large Abstract}

\begin{justify}
	The practical lab report \textit{``Operating Systems''} is the original and unmodified content submitted by \textit{Kushagra Lakhwani} (Roll No. 2021UCI8036).

	The report is submitted to \textit{Dr. Manoj Kumar} Department of Computer Science and Engineering, NSUT, Delhi, for the partial fulfillment of the requirements of the course \textit{``Operating Systems''} (CICPC09).
\end{justify}

\pagebreak


\thispagestyle{empty}
\tableofcontents
\newpage

\section{IPv4 Address Conversion}
\label{sec:ipv4}
\subsection{Objective}
To convert a binary IP address into dotted decimal and vice versa.
\subsection{Source Code}
\inputminted[firstline=5, lastline=39, fontsize=\footnotesize]{cpp}{code/ipv4.cpp}
\subsection{Output}

\subsubsection{Binary to dotted decimal IP address}
\begin{lstlisting}[style=output]
$ ./ipv4
1. Binary to dotted decimal IP address
2. Dotted decimal to binary IP address
Enter your choice: 1
Enter binary IP address (32 bits): 11000000101010000000000100000001
Dotted Decimal IP address: 192.168.1.1
\end{lstlisting}

\subsubsection{Dotted decimal to binary IP address}

\begin{lstlisting}[style=output]
$ ./ipv4
1. Binary to dotted decimal IP address
2. Dotted decimal to binary IP address
Enter your choice: 2
Enter dotted decimal IP address (e.g., 192.168.1.1): 203.128.56.2
Binary IP address: 11001011100000000011100000000010
\end{lstlisting}

\pagebreak

\section{IP Address Classes}
\label{sec:ipclass}
\subsection{Objective}
To identify the class of an IP address.
\subsection{Theory}
In IPv4, IP addresses are divided into five classes: A, B, C, D, and E. Each
class has its own range of valid IP addresses and is used for specific
purposes.

\begin{enumerate}[label=\textbf{Class \Alph*:}, leftmargin=2cm]
	\item \begin{itemize}
		      \item Range: 1.0.0.0 to 126.255.255.255
		      \item Subnet Mask: 255.0.0.0
		      \item Address Allocation: Class A addresses are typically used by large organizations and corporations. They can support a very large number of hosts on a single network.
	      \end{itemize}
	      
	\item \begin{itemize}
		      \item Range: 128.0.0.0 to 191.255.255.255
		      \item Subnet Mask: 255.255.0.0
		      \item Address Allocation: Class B addresses are used by medium-sized organizations. They offer a moderate number of network and host addresses.
	      \end{itemize}
	      
	\item \begin{itemize}
		      \item Range: 192.0.0.0 to 223.255.255.255
		      \item Subnet Mask: 255.255.255.0
		      \item Address Allocation: Class C addresses are commonly used by small organizations and businesses. They provide a limited number of network addresses but a larger number of host addresses.
	      \end{itemize}
	      
	\item \begin{itemize}
		      \item Range: 224.0.0.0 to 239.255.255.255
		      \item Address Allocation: Class D addresses are reserved for multicast groups and are not used for traditional unicast communication. They are used for one-to-many or many-to-many communication.
	      \end{itemize}
	      
	\item \begin{itemize}
		      \item Range: 240.0.0.0 to 255.255.255.255
		      \item Address Allocation: Class E addresses are reserved for experimental or research purposes and are not typically used in public networks. They are reserved for future use and not intended for general use.
	      \end{itemize}
\end{enumerate}


\subsection{Source Code}
\inputminted[firstline=5, lastline=25, fontsize=\footnotesize]{cpp}{code/ipv4class.cpp}
\subsection{Output}
\begin{lstlisting}[style=output]
$ ./ipv4class
Enter an IPv4 address: 192.168.1.1
Class: C
\end{lstlisting}

\pagebreak

\section{Bellman-Ford Algorithm}
\label{sec:Bellman-Ford Algorithm}

\subsection{Objective}
To implement the Bellman-Ford algorithm to find the shortest path
in a weighted graph.

\subsection{Theory}
The Bellman-Ford algorithm is used to find the shortest paths from a 
single source vertex to all other vertices in a weighted graph, even when the
graph contains negative weight edges. While it's not the most efficient
algorithm for all cases (especially for graphs with non-negative weights, where
Dijkstra's algorithm is typically faster),

\subsection{Source Code}
\inputminted[firstline=6, lastline=55, fontsize=\footnotesize]{cpp}{code/bellmanford.cpp}

\subsection{Output}
\begin{lstlisting}[style=output]
Enter the number of vertices and edges: 3 4
Enter edge 1 (source, destination, weight): 0 1 5
Enter edge 2 (source, destination, weight): 1 0 3
Enter edge 3 (source, destination, weight): 1 2 -1
Enter edge 4 (source, destination, weight): 2 0 1
Enter the source vertex: 2
Vertex  Distance from Source
0       1
1       6
2       0
\end{lstlisting}

\end{document}
